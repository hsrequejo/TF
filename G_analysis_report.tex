% Options for packages loaded elsewhere
\PassOptionsToPackage{unicode}{hyperref}
\PassOptionsToPackage{hyphens}{url}
\PassOptionsToPackage{dvipsnames,svgnames*,x11names*}{xcolor}
%
\documentclass[
  12pt,
]{article}
\usepackage{lmodern}
\usepackage{amssymb,amsmath}
\usepackage{ifxetex,ifluatex}
\ifnum 0\ifxetex 1\fi\ifluatex 1\fi=0 % if pdftex
  \usepackage[T1]{fontenc}
  \usepackage[utf8]{inputenc}
  \usepackage{textcomp} % provide euro and other symbols
\else % if luatex or xetex
  \usepackage{unicode-math}
  \defaultfontfeatures{Scale=MatchLowercase}
  \defaultfontfeatures[\rmfamily]{Ligatures=TeX,Scale=1}
\fi
% Use upquote if available, for straight quotes in verbatim environments
\IfFileExists{upquote.sty}{\usepackage{upquote}}{}
\IfFileExists{microtype.sty}{% use microtype if available
  \usepackage[]{microtype}
  \UseMicrotypeSet[protrusion]{basicmath} % disable protrusion for tt fonts
}{}
\makeatletter
\@ifundefined{KOMAClassName}{% if non-KOMA class
  \IfFileExists{parskip.sty}{%
    \usepackage{parskip}
  }{% else
    \setlength{\parindent}{0pt}
    \setlength{\parskip}{6pt plus 2pt minus 1pt}}
}{% if KOMA class
  \KOMAoptions{parskip=half}}
\makeatother
\usepackage{xcolor}
\IfFileExists{xurl.sty}{\usepackage{xurl}}{} % add URL line breaks if available
\IfFileExists{bookmark.sty}{\usepackage{bookmark}}{\usepackage{hyperref}}
\hypersetup{
  pdftitle={G Analysis Report},
  pdfauthor={Henrique S Requejo},
  colorlinks=true,
  linkcolor=black,
  filecolor=Maroon,
  citecolor=Blue,
  urlcolor=blue,
  pdfcreator={LaTeX via pandoc}}
\urlstyle{same} % disable monospaced font for URLs
\usepackage[margin=1in]{geometry}
\usepackage{graphicx,grffile}
\makeatletter
\def\maxwidth{\ifdim\Gin@nat@width>\linewidth\linewidth\else\Gin@nat@width\fi}
\def\maxheight{\ifdim\Gin@nat@height>\textheight\textheight\else\Gin@nat@height\fi}
\makeatother
% Scale images if necessary, so that they will not overflow the page
% margins by default, and it is still possible to overwrite the defaults
% using explicit options in \includegraphics[width, height, ...]{}
\setkeys{Gin}{width=\maxwidth,height=\maxheight,keepaspectratio}
% Set default figure placement to htbp
\makeatletter
\def\fps@figure{htbp}
\makeatother
\setlength{\emergencystretch}{3em} % prevent overfull lines
\providecommand{\tightlist}{%
  \setlength{\itemsep}{0pt}\setlength{\parskip}{0pt}}
\setcounter{secnumdepth}{5}
\usepackage[utf8]{inputenc}
\usepackage{subfig}
\usepackage{float}
\usepackage{amsbsy}
\usepackage[portuguese]{babel}

\usepackage{booktabs}
\usepackage{longtable}
\usepackage{array}
\usepackage{multirow}
\usepackage{wrapfig}
\usepackage{float}
\usepackage{colortbl}
\usepackage{pdflscape}
\usepackage{tabu}
\usepackage{threeparttable}
\usepackage{threeparttablex}
\usepackage[normalem]{ulem}
\usepackage{makecell}
\usepackage{xcolor}
\usepackage{setspace}
\usepackage{indentfirst}
\usepackage[font=small,labelfont=bf]{caption}

\newcommand*{\secref}[1]{Section~\ref{#1}}

\floatplacement{figure}{H}
\onehalfspacing
\graphicspath{ {./images/} }
\setlength{\parindent}{2em}
\AtBeginDocument{\let\maketitle\relax}
\usepackage{booktabs}
\usepackage{longtable}
\usepackage{array}
\usepackage{multirow}
\usepackage{wrapfig}
\usepackage{float}
\usepackage{colortbl}
\usepackage{pdflscape}
\usepackage{tabu}
\usepackage{threeparttable}
\usepackage{threeparttablex}
\usepackage[normalem]{ulem}
\usepackage{makecell}

\title{G Analysis Report}
\author{Henrique S Requejo}
\date{}

\begin{document}
\maketitle

\pagenumbering{gobble}

\pagebreak

\tableofcontents

\pagebreak

\pagenumbering{arabic}

\hypertarget{visuxe3o-geral-da-rede-ihs-70-400}{%
\section{Visão geral da rede
IHS-70-400}\label{visuxe3o-geral-da-rede-ihs-70-400}}

A rede IHS-70-400 possui 2 camadas (layer2 e layer1), 470 nós e 1522
conexões. A tabela \ref{tab:1} mostra o resumo das propriedades da rede.
A figura \ref{fig:fig1} apresenta uma visão geral da rede.

\begin{table}[!h]

\caption{\label{tab:Tabela_Prop_Rede}\label{tab:1}Propriedades da rede  IHS-70-400}
\centering
\begin{tabular}[t]{ll}
\toprule
Propriedade & Valor\\
\midrule
\cellcolor{gray!6}{Número de Camadas} & \cellcolor{gray!6}{2}\\
Tipo de conexões & layer2  e  layer1\\
\cellcolor{gray!6}{Número de nós} & \cellcolor{gray!6}{470}\\
Número de conexões & 1522\\
\bottomrule
\end{tabular}
\end{table}

\begin{figure}[H]

{\centering \includegraphics{G_analysis_report_files/figure-latex/Figura_Overview-1} 

}

\caption{\label{fig:fig1}Visão geral da rede  IHS-70-400 .}\label{fig:Figura_Overview}
\end{figure}

\hypertarget{resultados-da-rede-ihs-70-400}{%
\section{Resultados da rede
IHS-70-400}\label{resultados-da-rede-ihs-70-400}}

\hypertarget{distribuiuxe7uxe3o-de-mathbfg_norm}{%
\subsection{\texorpdfstring{Distribuição de
\(\mathbf{G_{norm}}\)}{Distribuição de \textbackslash mathbf\{G\_\{norm\}\}}}\label{distribuiuxe7uxe3o-de-mathbfg_norm}}

A variável \(G\) foi calculada para 10 partições de \(\omega\), ou seja,
o tamanho do passo dado dentro de \(\omega\) foi de 0.1. O processo foi
repetido para 16 partições de \(\gamma\), com \(\gamma\) começando em
0.25, com passos de 0.25 até um \(\gamma\) máximo de 4. O cálculo de
\(\overline{G}\) foi feito usando 100 iterações. A tabela \ref{tab:2}
resume os parâmetros de execução do código e a figura \ref{fig:fig2}
mostra a distribuição dos valores de \(G_{norm}\) médio obtidos.

\begin{table}[!h]

\caption{\label{tab:Parametros_Execucao}\label{tab:2}Parâmetros de execucao}
\centering
\begin{tabular}[t]{lr}
\toprule
Parâmetro & Valor\\
\midrule
\cellcolor{gray!6}{Iterações} & \cellcolor{gray!6}{100}\\
Partições de omega & 10\\
\bottomrule
\end{tabular}
\end{table}

\begin{figure}[H]

{\centering \includegraphics{G_analysis_report_files/figure-latex/Distribuicao_Gnorm-1} 

}

\caption{\label{fig:fig2}Distribuição de $G_{norm}$ médio da rede  IHS-70-400 .}\label{fig:Distribuicao_Gnorm}
\end{figure}

\hypertarget{variauxe7uxe3o-de-mathbfoverlineg-por-boldsymbolomega}{%
\subsection{\texorpdfstring{Variação de \(\mathbf{\overline{G}}\) por
\(\boldsymbol{\omega}\)}{Variação de \textbackslash mathbf\{\textbackslash overline\{G\}\} por \textbackslash boldsymbol\{\textbackslash omega\}}}\label{variauxe7uxe3o-de-mathbfoverlineg-por-boldsymbolomega}}

Como temos dados em 3 dimensões (\(\overline{G}\), \(\omega\),
\(\gamma\)) temos algumas formas diferentes para apresentar os valores
de \(\overline{G}\) em relação a \(\omega\) e \(\gamma\), não sei dizer
se devemos usar uma delas, as três ou alguma outra. A figura
\ref{fig:1a} mostra curvas de decaimento de \(\overline{G}\) por
\(\omega\) para diferentes nós com diferentes valores de \(G_{norm}\) e
para diferentes valores de \(\gamma\). A figura \ref{fig:1a.1} mostra a
superfície 3D formada por \(\overline{G}\) em relação a \(\omega\) e
\(\gamma\). A figura \ref{fig:1a.2} mostra a mesma superfície da figura
\ref{fig:1a.1} mas no formato de mapa de calor.

\begin{figure}[H]

{\centering \subfloat[resource_114 . $G_{norm} =$ 1.368\label{fig:decaimentos_ilustrativos-1}]{\includegraphics[width=.49\linewidth]{G_analysis_report_files/figure-latex/decaimentos_ilustrativos-1} }\subfloat[resource_97 . $G_{norm} =$ 1.363\label{fig:decaimentos_ilustrativos-2}]{\includegraphics[width=.49\linewidth]{G_analysis_report_files/figure-latex/decaimentos_ilustrativos-2} }\newline\subfloat[resource_381 . $G_{norm} =$ 1.003\label{fig:decaimentos_ilustrativos-3}]{\includegraphics[width=.49\linewidth]{G_analysis_report_files/figure-latex/decaimentos_ilustrativos-3} }\subfloat[resource_38 . $G_{norm} =$ 0.849\label{fig:decaimentos_ilustrativos-4}]{\includegraphics[width=.49\linewidth]{G_analysis_report_files/figure-latex/decaimentos_ilustrativos-4} }

}

\caption{\label{fig:1a}Exemplos de curvas do decaimento de $\overline{G}$ em relação a $\omega$ e $\gamma$ para diferentes valores de $\gamma$ da rede  IHS-70-400 . (a) Curvas de $\overline{G}$ da especie com maior valor de $G_{norm}$ da rede. (b) Segundo maior valor de $G_{norm}$. (c) Valor de $G_{norm}$ mais proximo da média geral da rede. (d) Curvas de $\overline{G}$ referente a uma espécie com valor de $G_{norm}$ abaixo da média da rede.}\label{fig:decaimentos_ilustrativos}
\end{figure}

\begin{figure}[H]

{\centering \subfloat[resource_114 . $G_{norm} =$ 1.368\label{fig:decaimentos_ilustrativos_sup_3D-1}]{\includegraphics[width=.49\linewidth]{G_analysis_report_files/figure-latex/decaimentos_ilustrativos_sup_3D-1} }\subfloat[resource_97 . $G_{norm} =$ 1.363\label{fig:decaimentos_ilustrativos_sup_3D-2}]{\includegraphics[width=.49\linewidth]{G_analysis_report_files/figure-latex/decaimentos_ilustrativos_sup_3D-2} }\newline\subfloat[resource_381 . $G_{norm} =$ 1.003\label{fig:decaimentos_ilustrativos_sup_3D-3}]{\includegraphics[width=.49\linewidth]{G_analysis_report_files/figure-latex/decaimentos_ilustrativos_sup_3D-3} }\subfloat[resource_38 . $G_{norm} =$ 0.849\label{fig:decaimentos_ilustrativos_sup_3D-4}]{\includegraphics[width=.49\linewidth]{G_analysis_report_files/figure-latex/decaimentos_ilustrativos_sup_3D-4} }

}

\caption{\label{fig:1a.1}Exemplos de superfícies do decaimento de $\overline{G}$ em relação a $\omega$ e $\gamma$ para diferentes valores de $\gamma$ da rede  IHS-70-400 . (a) Superfície de $\overline{G}$ da especie com maior valor de $G_{norm}$ da rede. (b) Segundo maior valor de $G_{norm}$. (c) Valor de $G_{norm}$ mais proximo da média geral da rede. (d) Superfície de $\overline{G}$ referente a uma espécie com valor de $G_{norm}$ abaixo da média da rede.}\label{fig:decaimentos_ilustrativos_sup_3D}
\end{figure}

\begin{figure}[H]

{\centering \subfloat[resource_114 . $G_{norm} =$ 1.368\label{fig:decaimentos_ilustrativos_heatmap-1}]{\includegraphics[width=.49\linewidth]{G_analysis_report_files/figure-latex/decaimentos_ilustrativos_heatmap-1} }\subfloat[resource_97 . $G_{norm} =$ 1.363\label{fig:decaimentos_ilustrativos_heatmap-2}]{\includegraphics[width=.49\linewidth]{G_analysis_report_files/figure-latex/decaimentos_ilustrativos_heatmap-2} }\newline\subfloat[resource_381 . $G_{norm} =$ 1.003\label{fig:decaimentos_ilustrativos_heatmap-3}]{\includegraphics[width=.49\linewidth]{G_analysis_report_files/figure-latex/decaimentos_ilustrativos_heatmap-3} }\subfloat[resource_38 . $G_{norm} =$ 0.849\label{fig:decaimentos_ilustrativos_heatmap-4}]{\includegraphics[width=.49\linewidth]{G_analysis_report_files/figure-latex/decaimentos_ilustrativos_heatmap-4} }

}

\caption{\label{fig:1a.2}Exemplos de mapas de calor do decaimento de $\overline{G}$ em relação a $\omega$ e $\gamma$ para diferentes valores de $\gamma$ da rede  IHS-70-400 . (a) Mapa de calor de $\overline{G}$ da especie com maior valor de $G_{norm}$ da rede. (b) Segundo maior valor de $G_{norm}$. (c) Valor de $G_{norm}$ mais proximo da média geral da rede. (d) Mapa de calor de $\overline{G}$ referente a uma espécie com valor de $G_{norm}$ abaixo da média da rede.}\label{fig:decaimentos_ilustrativos_heatmap}
\end{figure}

\end{document}
